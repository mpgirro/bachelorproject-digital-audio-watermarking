\chapter{Methode}
\label{ch:methode}

In Kapitel \ref{ch:theorie} wurde das mathematische Modell beschrieben, mit welchem ein Bit geschrieben werden kann!!

(ein Signal so ver\"andert werden kann, dass eine eingebettete Sequenz bin\"arer Daten eindeutig rekonstruiert werden kann.)

Ein Signal S wird mit Daten D angereichert, das ist dann S'. S' wird übertragen und durch die Übertragung zu S'' verändert. Nun soll aus S'' wieder D rekonstruiert werden.

Um dieses theoretische Wissen nun in einer Applikation praktisch nutzen zu k\"onnen, muss der Algorithmus in ein Framework eingebettet werden. 

\section{Architektur}

Prinzipiell zerteil sich in mehrere phasen, blockschaltbilder, etc

\section{Watermark Implantierung}
\label{sec:embedding}

\subsection{Synchronisations-Codes}

wozu sync codes ganz allgemein

\subsection{Fehlerkorrekturverfahren}
\label{sec:errorcorrection}

Error detection and correction

\subsubsection{BCH-Codes}

\subsubsection{RS-Codes}

\subsubsection{Turbo-Codes}

\subsection{Datenstrukturen und Protokoll}
\label{sec:protokoll}

\subsection{Qualitätskontrolle und ODG }

\subsubsection{Signal-Rauschabstand}

\subsubsection{Objective Difference Grade}

\paragraph{PQevalAudio}

\paragraph{EAQUAL}

\paragraph{peaqb}


\section{Watermark Extrahierung}
\label{sec:extraction}


\subsection{Resynchronisaton und Interpolation}

\subsection{Synchronisations-Code Erkennung}

\subsubsection{Autokorrelation und Barker-Codes}
\label{sec:barkercode}

vorteil der barker codes, autocorrelation

\subsection{Datenextrahierung}



