\chapter*{Kurzfassung}

Während digitales Audio-Watermarking einen an sich viel beachteten Bereich darstellt, beschränkt sich die Forschung dennoch hauptsächlich auf Anwendungsgebiete rund um Urheberrechtsschutz. Eine Ausrichtung auf offen eingebrachte Information ist kaum im Fokus der Entwicklung. Doch gibt es durchaus Anwendungsfälle in denen Zusatzinformationen in Audiosignalen einen erheblichen Mehrwert bedeuten können. Vor allem bei der Digital/Analog Wandlung gehen in der Regel sämtliche Metadaten eines Audiosignals verloren, da diese nicht in akustische Signale mit überführt werden.
Hier können Watermarks Abhilfe schaffen. Auch könnten sie genutzt werden um in Radio- oder Fernsehübertragungen gezielt zusätzliche Informationen einzubauen, welche auch von anderen Geräten aus dem ursprünglichen Empfangsgerät verarbeitet werden könnten.

Diese Arbeit beschäftigt sich mit der Implementierung einer vorhandenen Audio-Water\-marking Methode welche für die D/A-A/D Wandlung geeignet sein soll. Weiters wurde für deren Verwendung ein Framework entwickelt, welches mittels eines eigens entwickelten Übertragungsprotokoll unter Verwendung von Synchronisations-Codes und Fehlerkorrekturverfahren einen definierten Übertragungskanal bereitstellt. 